\documentclass{article}

\usepackage{listings}
\usepackage{hyperref}
\usepackage{color}
\usepackage{courier}
\usepackage[left=20px,right=20px,top=10px,bottom=10px]{geometry}

\lstset{basicstyle=\footnotesize\ttfamily,breaklines=true}

\author{J\"orn Neumeyer}

\title{Understanding C/C++ pointer arithmetic and memory allocation}

\definecolor{darkturquoise}{rgb}{0.0, 0.81, 0.82}
\definecolor{gray}{rgb}{0.5, 0.5, 0.5}
\definecolor{bostonuniversityred}{rgb}{0.8, 0.0, 0.0}
\definecolor{orangepeel}{rgb}{1.0, 0.62, 0.0}

\lstset{
  basicstyle=\ttfamily,
  language={C++},
  numbers=left,
  numbersep=5pt,
  title=\lstname,
  stringstyle=\color{red},
  showspaces=false,
  showstringspaces=false,
  morecomment=[l][\color{magenta}]{\#},
  directivestyle=\color{green},
  commentstyle=\color{gray},
  classoffset=0,
  % morekeywords={@property,@synthesize,super},
  keywordstyle=\color{blue},
  classoffset=1,
  morekeywords={NSString,Foo,NSAutoreleasePool,NSObject},
  keywordstyle=\color{orangepeel},
}

\begin{document}
  \maketitle
  \tableofcontents
  \section{Introduction}
  In this article I want to explain pointer arithmetic and memory allocation in C/C++.
  Sine this will not be a tutorial on C/C++ in general, but only on the mentioned sub-topics, you should have some knowledge in C/C++ or similar object-oriented languages.
  \section{A Simplified Recap On Memory Areas}
  In order to understand pointer arithmetic we should first have a look on the different memory areas in our program.
  Every program has dedicated virtual memory in the RAM which is (basically) divided into three parts: program, heap, and stack.
  The program memory simply holds binary instructions, so it is not interesting for us at the moment.
  \subsection{The Stack}
  The stack memory functions just like a standard stack collection: items are pushed to and popped from its top as needed.
  In the case of C, local variables are pushed to the stack as soon as they are initialized and popped after their scope has been left.
  To showcase this mechanism of pushing, I added an example in \textbf{example-stack.c}.
  The stack also has a limited size.
  These sizes can be found in a document of the Computer Science Department at New York University regarding \href{https://cs.nyu.edu/exact/core/doc/stackOverflow.txt}{Stack Overflow Problems}, in case you would like to know the exact numbers.
  Since the stack memory is chunk of memory already allocated for us, it is pretty fast to place variables in it.
  \\So, the main takeaway is that memory we allocated on the stack is automatically freed for us and pretty fast to allocate.
  \subsection{The Heap}
  On the other side we have the heap memory.
  In this area things aren't quite as ordered and automated as they are on the stack.
  The main differences lies in the way of allocation and lifetime of heap memory.
  So, let's have a look at virtual memory again.
  An application's virtual memory size something like 2-3GB.
  Now, that memory block is not already allocated for us like the stack, but it is more or less the amount of memory we can allocate on the heap.
  Allocation also works different from its stack variant.
  Later on we'll be using the specific C/C++ functionality to actually allocate the memory.
  But the way it works in general will stay the same: as soon as you request a set amount of memory inside your program, the operating system is going to allocate that amount of memory on the heap and return a pointer to the beginning of the freshly allocated memory.
  There is a little more going on in the background, but we'll skip that for simplicity.
  Since the memory has been allocated outside the stack, it will stay there until it is freed explicitly.
  Therefore, that memory will stay valid even after leaving the scope in which it has been allocated.
  \\So, the advantage of heap allocated memory over the stack one is the amount of memory we are able to allocate and the fact that this memory lives outside the scope of its creation.
  However, the main disadvantage is the long time it takes for the operating system to allocate the memory, requested by your program.
  \section{Working With Pointers}
  In the previous heap section I already mentioned that the operating system will return a pointer to the beginning of heap allocated memory.
  So, it seems that pointers play a pretty big role when it comes down to handling any sort of memory.
  But why do we even have pointers and what are they used for?
  \subsection{Referencing Memory}
  In C/C++ you often used reference variables instead of directly using their values.
  To simplify, references are used whenever you want to operate on memory (e.g. a variable) with an unknown address.
  Although unknown seems a little strange, it is the best word to describe how it actually is.
  Let me give you an example.
  Let's assume we want to read a number from \textbf{stdin} (the console) into a variable as shown in the code below.
  \lstinputlisting{example-stdin.c}
  The function \textbf{scanf} can read a value from \textbf{stdin} for us.
  Since it is common for functions in C to return a success code, they typically write their result to so called \underline{output parameters}.
  But functions like \textbf{scanf} do not have a clue to what location they should write their result.
  So, we have to tell the function to what location it should write the data with a pointer.
  That's basically all a pointer is: some location in memory.
  As shown in line \textbf{line 7}, we can a pointer to a variable (its memory address) by using the ampersand (\&) operator.
  \subsection{Using Referenced Memory}
  Let's say we want to write something simple which could go wrong.
  
  \section{Copyright}
  Copyright \textcopyright \enspace2019 J\"orn Neumeyer\\
  Unless specified otherwise, the content of this article is licensed under the \href{https://creativecommons.org/licenses/by-sa/4.0/legalcode}{Creative Commons Attribution-ShareAlike 4.0 International Public License}.
  You may modify, copy, distribute, and display this material, but you must give credit to the original authors. Please see the license for details.
  \section{Source File Content}
  \lstinputlisting{example-stack.c}
\end{document}
