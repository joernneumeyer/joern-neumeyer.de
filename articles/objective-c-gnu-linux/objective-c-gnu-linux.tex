\documentclass{article}

\usepackage{listings}
\usepackage{hyperref}
\usepackage{color}
\usepackage{courier}
\usepackage[left=20px,right=20px,top=10px,bottom=10px]{geometry}

\lstset{basicstyle=\footnotesize\ttfamily,breaklines=true}

\title{How to compile Objective-C code on GNU/Linux based operating systems with GNUstep}
\author{J\"orn Neumeyer}

\definecolor{darkturquoise}{rgb}{0.0, 0.81, 0.82}
\definecolor{gray}{rgb}{0.5, 0.5, 0.5}
\definecolor{bostonuniversityred}{rgb}{0.8, 0.0, 0.0}
\definecolor{orangepeel}{rgb}{1.0, 0.62, 0.0}

\lstset{
  basicstyle=\ttfamily,
  language={[Objective]C},
  numbers=left,
  numbersep=5pt,
  title=\lstname,
  stringstyle=\color{red},
  showspaces=false,
  showstringspaces=false,
  morecomment=[l][\color{magenta}]{\#},
  directivestyle=\color{green},
  commentstyle=\color{gray},
  classoffset=0,
  morekeywords={@property,@synthesize,super},
  keywordstyle=\color{blue},
  classoffset=1,
  morekeywords={NSString,Foo,NSAutoreleasePool,NSObject},
  keywordstyle=\color{orangepeel},
  classoffset=2,
  morekeywords={NSLog},
  keywordstyle=\color{darkturquoise},
  classoffset=3,
  morekeywords={new},
  keywordstyle=\color{bostonuniversityred}
}

\begin{document}
  \maketitle
  \tableofcontents
  \section{Required Packages}
  In order to compile Objective-C on GNU/Linux, we need a couple of packages:
  \begin{itemize}
    \item clang
    \item gnustep-make
    \item gnustep-devel
    \item libobjc2 (from \url{https://github.com/gnustep/libobjc2})
    \item optionally: libobjc-8-dev (see building section)
  \end{itemize}
  \section{Installation}
  Installation is pretty straight forward.
  First, install clang and the GNUstep packages via your package manager of choice. In my case it's apt.
  \begin{center}
    sudo apt install clang gnustep-make gnustep-devel
  \end{center}
  The installation of \textbf{libobjc2} is also easy.
  Just download the zipped source code of the latest release and extract it.
  After that, we just have to link \textbf{libobjc2} properly, so it can be used.
  In this example, the ZIP has been extracted to the folder \textbf{libobjc2} in my personal Downloads.
  \begin{center}
    sudo ln -s /home/foo/Downloads/libobjc2/objc /usr/include/objc
  \end{center}
  \section{Sample Code}
  The rest of this article will be based on the following source code files:
  \begin{itemize}
    \item main.m - The main file of the executable
    \item Foo.h - The header of the Foo library we build
    \item Foo.m - The implementation of the Foo library
  \end{itemize}
  The actual source code is appended at the end of this article.
  \newpage
  \subsection{Boilerplate Code}
  There is a little bit of boilerplate code in the main function.
  Namely the creation of an \textbf{NSAutoreleasePool} instance and the call of its \textbf{drain} method.
  \begin{lstlisting}
    NSAutoreleasePool* pool = NSAutoreleasePool.new; // line 5
    [pool drain]; // line 11
  \end{lstlisting}
  These lines can also be omitted, but that will lead to a large number of logging in your console as you run your code.
  \section{Building}
  It does not take much to compile an Objective-C file.
  Using clang, the command looks like the following:
  \begin{lstlisting}[language=bash]
    clang $(gnustep-config --objc-flags) $(gnustep-config --base-libs) main.m Foo.m
  \end{lstlisting}
  In case the build process fails because the compiler could not find \textbf{-lobjc}, try installing \textbf{libobjc-8-dev}.
  That should fix the missing library error.
  \subsection{Building with make}
  In my experience building with make does not work - sort of.
  As seen in the above section, we retrieve a bunch of compiler flags through gnustep-config.
  For some reason gnustep-config would not produce any output when run inside a Makefile.
  Therefore, I use a bash helper to retrieve the compiler flags and pass them to my Makefile.
  So, the important line of my build helper - build.sh - looks like this:
  \begin{lstlisting}[language=bash]
    make FLAGS="$(gnustep-config --objc-flags) $(gnustep-config --base-libs)"
  \end{lstlisting}
  Now we can easily consume the passed FLAGS variable in our Makefile:
  \begin{lstlisting}[language=make]
    main: main.m Foo.o
    clang $(FLAGS) main.m Foo.o -o main

    foo: Foo.h Foo.m
    clang $(FLAGS) Foo.m -c -o Foo.o
  \end{lstlisting}
  After enabling script execution - chmod +x build.sh - we can run our build script via ./build.sh and our code compiles successfully via make.
  \section{Copyright}
  Copyright \textcopyright \enspace2019 J\"orn Neumeyer\\
  Unless specified otherwise, the content of this article is licensed under the \href{https://creativecommons.org/licenses/by-sa/4.0/legalcode}{Creative Commons Attribution-ShareAlike 4.0 International Public License}.
  You may modify, copy, distribute, and display this material, but you must give credit to the original authors. Please see the license for details.
  \newpage
  \section{Source File Content}
  \lstinputlisting[frame=single,language=make]{Makefile}
  \lstinputlisting[frame=single,language=bash]{build.sh}
  \lstinputlisting[frame=single]{main.m}
  \lstinputlisting[frame=single]{Foo.h}
  \newpage
  \lstinputlisting[frame=single]{Foo.m}
\end{document}
